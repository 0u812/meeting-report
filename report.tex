\documentclass[journal,transmag]{IEEEtran}

% ..........................................................................
% Packages, configuration settings, and macro definitions.

\usepackage[pdftex]{graphicx}
\graphicspath{figures}
\DeclareGraphicsExtensions{.pdf,.jpeg,.png}

\usepackage[pdftex,rgb,dvipsnames,svgnames,hyperref,table]{xcolor}

\usepackage[pdftex,breaklinks=true,colorlinks=true,
  bookmarks=false,pdfhighlight=/O,
  urlcolor=darkblue,citecolor=darkred,linkcolor=darkblue]{hyperref}

\usepackage[cmex10]{amsmath}
\interdisplaylinepenalty=2500

\usepackage{amssymb}
\usepackage{amsfonts}
\usepackage{multicol}
\usepackage{enumitem}
\usepackage{accsupp}
\usepackage{array}
\usepackage[caption=false,font=footnotesize]{subfig}
\usepackage{booktabs}
\usepackage{xspace}
\usepackage{soul}
\usepackage{url}
\usepackage{natbib}
\usepackage{hyphenat}
\usepackage[english]{babel}
\usepackage{dblfloatfix}

% Correct bad hyphenation here
\hyphenation{op-tical net-works semi-conduc-tor}

% ..........................................................................
% Body.

\begin{document}

\markboth{IEEE Transactions on Biomedical Engineering}%
{Whole Cell Summer School Meeting Report}

\title{Whole Cell Summer School}

\author{\IEEEauthorblockN{%
Dagmar Waltemath\IEEEauthorrefmark{1},
Falk Schreiber\IEEEauthorrefmark{2}, and
Other People}%
\IEEEauthorblockA{\IEEEauthorrefmark{1}University of Rostock}
\IEEEauthorblockA{\IEEEauthorrefmark{2}Monash University}%
\thanks{Corresponding author email: dagmar.waltemath@uni-rostock.de}}

\IEEEtitleabstractindextext{%

\begin{abstract}
The abstract goes here.
\end{abstract}

% Note that keywords are not normally used for peerreview papers.
\begin{IEEEkeywords}
Systems Biology, Simulation, Computational Modeling, Standards
\end{IEEEkeywords}
%
}

\maketitle
\IEEEdisplaynontitleabstractindextext
\IEEEpeerreviewmaketitle


\section{Introduction}
% The very first letter is a 2 line initial drop letter followed
% by the rest of the first word in caps.
% 
% form to use if the first word consists of a single letter:
% \IEEEPARstart{A}{demo} file is ....
% 
% form to use if you need the single drop letter followed by
% normal text (unknown if ever used by IEEE):
% \IEEEPARstart{A}{}demo file is ....
% 
% Some journals put the first two words in caps:
% \IEEEPARstart{T}{his demo} file is ....
% 
% Here we have the typical use of a "T" for an initial drop letter
% and "HIS" in caps to complete the first word.
\IEEEPARstart{T}{his} is a report of the VW Whole Cell Summer School held in Rostock during April, 2015.


\subsection{Subsections}

\subsection{Subsections}

\section{Conclusion}
The conclusion goes here.

\section*{Acknowledgment}

The authors would like to thank...

% Can use something like this to put references on a page
% by themselves when using endfloat and the captionsoff option.
\ifCLASSOPTIONcaptionsoff
  \newpage
\fi

\bibliographystyle{IEEEtran}
\bibliography{IEEEabrv,report}


% biography section
% 
% If you have an EPS/PDF photo (graphicx package needed) extra braces are
% needed around the contents of the optional argument to biography to prevent
% the LaTeX parser from getting confused when it sees the complicated
% \includegraphics command within an optional argument. (You could create
% your own custom macro containing the \includegraphics command to make things
% simpler here.)
%\begin{IEEEbiography}[{\includegraphics[width=1in,height=1.25in,clip,keepaspectratio]{mshell}}]{Michael Shell}
% or if you just want to reserve a space for a photo:

% \begin{IEEEbiography}{Michael Shell}
% Biography text here.
% \end{IEEEbiography}

% if you will not have a photo at all:
% \begin{IEEEbiographynophoto}{John Doe}
% Biography text here.
% \end{IEEEbiographynophoto}

% insert where needed to balance the two columns on the last page with
% biographies
%\newpage

% \begin{IEEEbiographynophoto}{Jane Doe}
% Biography text here.
% \end{IEEEbiographynophoto}

% You can push biographies down or up by placing
% a \vfill before or after them. The appropriate
% use of \vfill depends on what kind of text is
% on the last page and whether or not the columns
% are being equalized.

%\vfill

% Can be used to pull up biographies so that the bottom of the last one
% is flush with the other column.
%\enlargethispage{-5in}



% ..........................................................................
% End.

\end{document}
